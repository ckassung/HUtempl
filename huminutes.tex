\documentclass[header,internal]{huminutes}

\title{%
  Abschlußsitzung}
\committee{%
  Wichtige Kommission}
\author{%
  Prof.\,Dr.\ Christian Kassung}
\date{%
  1.3.1999}
\whenmeeting{%
  30.\ April 2025}
\wheremeeting{%
  per Zoom}

\initiator{Prof.\,Dr.\ Christian Kassung}
\participant[present]{\LaTeX\ Benutzer}
\participant[information]{HU Angehörige}
\participant[absent]{Word Benutzer}



\begin{document}
\frontmatter


\section{Vorbereitung der \LaTeX\ Klasse für Protokolle}

Dies ist die \texttt{huminutes} Klasse. Diese Vorlage erlaubt das Schreiben von Protokollen entsprechend der Corporate Identity der Humboldt-Universtiät zu Berlin mit \LaTeX. Die Klasse übernimmt Funktionen aus der generischen \texttt{humain} Klasse; siehe die dortige Dokumentation für Anpassungen der Sprache und weitere benutzerspezifische Einstellungen.

Titel, Author und Datum werden mit den normalen \LaTeX\ Komandos \verb|\title|, \verb|\author| und \verb|\date| gesetzt. Weitere Optionen sind selbsterklärend und geben als Defaultwert die Information \texttt{set with \textbackslash command} über den Ausdruck an.

Die Sitzungsleitung \verb|\initiator| erscheint automatisch zuoberst in der Teilnehmerliste als anwesend. Weitere Teilnehmer können in der Präambel mit dem \verb|\participant| Kommando hinzugefügt werden, dem \emph{eines} der folgenden optionalen Argumente übergeben werden kann: \verb|present|, \verb|absent| und \verb|information|. Zum Beispiel:

\begin{verbatim}
  \participant[present]{\LaTeX\ users}
\end{verbatim}


\section{Sitzung abhalten}


\section{Protokoll schreiben}

Abschließend kann eine TODO-Liste über \verb|\tasklist| generiert werden:

\task{\LaTeX\ erlernen}{Word Benutzer}{ASAP}
\task{Diese Vorlage ausprobieren}{\TeX perts}{Jederzeit}
\task{Sich freuen}{\LaTeX\ Benutzer}{Ab jetzt}
\tasklist

\section{Kompilieren}

\end{document}